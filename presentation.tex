\documentclass{beamer}
\usepackage{graphicx}
\usetheme{Boadilla}
\usecolortheme{beetle}


\title{Group Therapy for the Type-Curious}
\subtitle{Theory \& Practice}
\author{Bruce C. Miller}
\institute{bm3719@gmail.com}
\date{\today}

\begin{document}

\begin{frame}
\titlepage
\end{frame}

\begin{frame}
\frametitle{Overview}

\begin{description}
\item[Type Theory]
\item[Type Systems]
\item[Static vs. Dynamic]
\item[Types in Clojure]
\end{description}

\end{frame}


\begin{frame}
  \frametitle{Purpose}

  \centerline{\includegraphics[scale=0.4]{img/transformation.png}}

\end{frame}


\begin{frame}
  \frametitle{Type Theory}

  Some History:
\begin{description}
\item[1902] Types are first proposed by Bertrand Russell as a solution to
  Russell's Paradox in Gottlob Frege's Naïve Set Theory.
\item[1940] Types are first applied to the programming language theory,
  combined with Alonzo Church's λ-calculus.
\item[1972] System F created.  Later to influence ML, Caml, Haskell.
\item[1972] Per Martin Löf Type Theory introduced.  Creates what's now known as
  dependent type theory, as used in Agda, Idris, Coq, Lean.
\item[2009] What is now known as Homotopy Type Theory introduced in a paper by
  Voevodsky.
\end{description}


\end{frame}

\begin{frame}
\frametitle{Type Systems}

From Pierce:

\vspace{20pt}
\textit{A type system is a tractable syntactic method for proving the absence of
  certain program behaviors by classifying phrases according to the kinds of
  values they compute.}

\end{frame}

\begin{frame}
\frametitle{Type Systems}

From Pierce:

\vspace{20pt}
\textit{A type system is a tractable syntactic method for proving the absence of
  certain program behaviors by classifying phrases according to the kinds of
  values they compute.}

\end{frame}

\begin{frame}
\frametitle{Type Systems}

From Pierce:

\vspace{20pt}
\textit{A type system is a tractable syntactic method for proving the absence of
  certain program behaviors by classifying phrases according to the kinds of
  values they compute.}

\end{frame}

\begin{frame}
\frametitle{Resources}

\end{frame}


\end{document}
