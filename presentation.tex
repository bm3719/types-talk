\documentclass{beamer}
\usepackage{graphicx}
\usetheme{Boadilla}
\usecolortheme{beetle}


\title{Group Therapy for the Type-Curious}
\subtitle{Theory \& Practice}
\author{Bruce C. Miller}
\institute{bm3719@gmail.com}
\date{\today}

\begin{document}

\begin{frame}
\titlepage
\end{frame}

\begin{frame}
  \frametitle{Purpose}
  \centerline{\includegraphics[scale=0.4]{img/transformation.png}}
\end{frame}

\begin{frame}
\frametitle{Overview}
\begin{description}
\item[Type Theory] Introduction and fundamentals
\item[Type Systems] Type theory applied to programming languages
\item[Static vs. Dynamic] And other dichotomies
\item[Types vs. Clojure] Begun, the Type Wars have
\end{description}
\end{frame}

\begin{frame}
  \frametitle{Type Theory}
  Some History:
\begin{description}
\item[1902] Types are first proposed by Bertrand Russell as a solution to
  Russell's Paradox in Cantor's Na{\"i}ve Set Theory.
\item[1940] Types are first applied to the programming language theory,
  combined with Alonzo Church's $\lambda$-calculus.
\item[1972] System F created.  Later to influence ML, Caml, Haskell.
\item[1972] Per Martin L{\"o}f Type Theory introduced.  Creates what's now
  known as dependent type theory, as used in Agda, Idris, Coq, Lean.
\item[2009] What is now known as Homotopy Type Theory introduced in a paper by
  Voevodsky.
\end{description}
\end{frame}

\begin{frame}
\frametitle{Type Systems}
From Pierce:
\vspace{20pt}

\textit{A type system is a tractable syntactic method for proving the absence of
  certain program behaviors by classifying phrases according to the kinds of
  values they compute.}

\end{frame}

\begin{frame}
\frametitle{Type Dichotomies}
\end{frame}

\begin{frame}
  \frametitle{Static vs. Dynamic Typing in Practice}
  Dynamic Types:
\begin{description}
\item[Brevity] Reduction in syntactic overhead.
\item[]
\end{description}
\end{frame}

\begin{frame}
\frametitle{Resources}
\begin{thebibliography}{9}

\bibitem{TTFP} R. Nedepelt, H. Geuvers, \emph{Type Theory and Formal Proof: An
    Introduction}, 1st Edition, Cambridge Univ. Press, Cambridge, MA, 2014.

\bibitem{TAPL} B. Pierce, \emph{Types and Programming Languages}, MIT
  Press, Cambridge, MA, 2002.

\bibitem{PandT} J. Girard, \emph{Proofs and Types}, Cambridge Tracts on
  Comp. Sci., Cambridge, MA, 1989.

\bibitem{ATTAPL} B. Pierce, \emph{Advanced Topics in Types and Programming
    Languages}, MIT Press, Cambridge, MA, 2004.


\end{thebibliography}
\end{frame}


\end{document}
